\input{lib/packs}
\LARGE
\begin{document}
\input{lib/macro}

Пример системы, траекторный аттрактор которой целиком лежит вне пространства траекторий,
и у которой при этом существует аттрактор в смысле динамической системы.

Введём в трёхмерном пространстве цилиндрическую систему координат
%TODO: ссылка?
$(\rho, \varphi, z)$, $\rho \in [0; \infty)$, $\varphi \in [0; 2\pi)$, $z \in (-\infty; \infty)$.
Для удобства записи будет полагать $\varphi = 0$ при $\rho = 0$.
Введём на трёхмерном пространстве некоторое произвольное отношение порядка.

Рассмотрим тор
$$
	T=\left\{
		(\rho, \varphi, z)
	\mid
		(\rho-2)^2 + z^2 = 1,
		\varphi \in [0; 2\pi)
	\right\}
$$
и его меридиан
$$
	M=\left\{
		(\rho, 0, z)
	\mid
		(\rho-2)^2 + z^2 = 1,
	\right\}.
$$
Зададим динамическую систему
%TODO: ссылка на зелёную книжку
операторами сдвига.
Пусть сначала сдвигаемая точка $(\rho, \varphi, z) \in T$.
Тогда положим
\begin{equation}\label{opsdviga_tor_in}
	S_t((\rho, \varphi, z)) =

\end{equation}




\end{document}
