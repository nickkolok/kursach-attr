\documentclass[a4paper,14pt]{report} %размер бумаги устанавливаем А4, шрифт 12пунктов
\usepackage[T2A]{fontenc}
\usepackage[utf8]{inputenc}
\usepackage[english,russian]{babel} %используем русский и английский языки с переносами
\usepackage{amssymb,amsfonts,amsmath,mathtext,cite,enumerate,float,amsthm} %подключаем нужные пакеты расширений
\usepackage[pdftex,unicode,colorlinks=true,linkcolor=blue]{hyperref}
\usepackage{indentfirst} % включить отступ у первого абзаца
\usepackage[dvips]{graphicx} %хотим вставлять рисунки?
\graphicspath{{illustr/}}%путь к рисункам

\makeatletter
\renewcommand{\@biblabel}[1]{#1.} % Заменяем библиографию с квадратных скобок на точку:
\makeatother %Смысл этих трёх строчек мне непонятен, но поверим "Запискам дебианщика"

\usepackage{geometry} % Меняем поля страницы. 
\geometry{left=1cm}% левое поле
\geometry{right=1cm}% правое поле
\geometry{top=1cm}% верхнее поле
\geometry{bottom=2cm}% нижнее поле

\renewcommand{\theenumi}{\arabic{enumi}}% Меняем везде перечисления на цифра.цифра
\renewcommand{\labelenumi}{\arabic{enumi}}% Меняем везде перечисления на цифра.цифра
\renewcommand{\theenumii}{.\arabic{enumii}}% Меняем везде перечисления на цифра.цифра
\renewcommand{\labelenumii}{\arabic{enumi}.\arabic{enumii}.}% Меняем везде перечисления на цифра.цифра
\renewcommand{\theenumiii}{.\arabic{enumiii}}% Меняем везде перечисления на цифра.цифра
\renewcommand{\labelenumiii}{\arabic{enumi}.\arabic{enumii}.\arabic{enumiii}.}% Меняем везде перечисления на цифра.цифра


\LARGE
\begin{document}
\newcommand{\pp}{Предположим противное}
%\newcommand{\pp}{{\LARП\!\!\!\!п~}}
\newcommand{\dokvo}{\paragraph{Доказательство.}}
\newcommand{\dokno}{\textbf {Доказано.}}
\newcommand{\neobh}{\paragraph{Необходимость.}}
\newcommand{\dost }{\paragraph{Достаточность.}}
\newcommand{\opred}{\paragraph{Определение.}}
\newcommand{\mnemo}{\paragraph{Мнемоника.}}
\newcommand{\N}{\mathbb{N}}
\newcommand{\Z}{\mathbb{Z}}
\newcommand{\Q}{\mathbb{Q}}
\newcommand{\R}{\mathbb{R}}
\newcommand{\one}[1]{\mathbb{I}_{#1}}
\renewcommand{\C}{\mathbb{C}}
\newcommand{\Beta}{B}%Костыль, а что поделать?
\newcommand{\Rn}{$\mathbb{R}^n~$}
\newcommand{\Rm}{$\mathbb{R}^m~$}
\renewcommand{\epsilon}{\varepsilon}
\renewcommand{\geq}{\geqslant}
\renewcommand{\leq}{\leqslant}
\newcommand{\fXR}{Пусть $X \subset \R, f:X \to \R$ }
\newcommand{\fXRx}{\fXR, $x_0$ - предельная точка $X$ }
\newcommand{\sgn}{\mathrm{sgn}~}
\newcommand{\nid}{\Leftrightarrow}
\newcommand{\intl}{\int\limits}
\newcommand{\suml}{\sum\limits}
\newcommand{\Models}{|\!\!\!=\!\!\!|}
\newcommand{\Rightleftarrow}{\Leftrightarrow}

\newcommand{\xI}{{\vec{\xi}}}%Костыль для тервера, очень уж там часто встречается
\newcommand{\calF}{\mathcal{F}}
\newcommand{\calB}{\mathcal{B}}
\newcommand{\GOFP}{$G \sim \left<\Omega,\calF,P\right>$}

\newenvironment{zamena}[1][c]{=\left<\begin{array}{#1}}{\end{array}\right>=}

\newtheorem{theorem}{Теорема}[section]
\newenvironment{teorema}[1][{}]{\begin{theorem}{#1}\upshape}{\end{theorem}}

\theoremstyle{definition}

\newtheorem{zamech}{Замечание}[section]
\newtheorem{primer}{Пример}[section]
\newtheorem{opr}{Определение}[section]

\newtheorem{sledstvie}{Следствие}[theorem]
\newtheorem{utverzhd}[theorem]{Утверждение}
\newtheorem{lemma}[theorem]{Лемма}

\long\def\comment{}


Курсовая работа


Непонятно: где брать образец титульного листа.


\opred
Траекторией (решением) дифференциального уравнения называется функция, при подстановке обращающая его в тождественное равенство на требуемом множестве значений аргумента.


Непонятно: чем может быть задано пространство траекторий, кроме как уравнением?
Полугруппой?
Ещё чем-то?


\opred
Пространством траекторий заданного дифференциального уравнений называется множество его решений.

Заметим, что пространство траекторий, хотя и называется пространством, может не иметь структуры линейного пространства:
сумма (в поточечном смысле) решений дифференициального уравнения вовсе не обязательно является его решением.

Рассмотрим, например, обыкновенное дифференциальное уравнение $y'(t) = 2t$.
Функции $f(t)=t^2$ и $g(t)=t^2 +1$ являются его решениями, в то время как их сумма $h(t) = f(t) + g(t) = 2t^2 + 1$ не является решением.


Необходимо: выяснить, подпространством какого пространства является пространство траекторий.
Предположительно $C_{[0;+\infty)}$.


Необходимо: определение: аттрактор.


Необходимо: определение: траекторный аттрактор.


Необходимо: определение: минимальный траекторный аттрактор.

В \cite{Kondratyev} (сразу после определения 4.5) указывается на смену терминологии в статье.
Правильно ли я понимаю, что следует пользоваться терминологией из 4 части?


Необходимо: определение: полугруппа.


\opred (\cite{Zelenaya}, стр. 118)
Оператором сдвига $T(h)$, $h\in\R$ называется оператор, который функции $f$ ставит в соответствие функцию $T(h)f$, такую, что
$$
T(h)f(t)=f(t+h)
$$


\opred (\cite{Zelenaya}, стр. 121)
Пространство траекторий $\mathcal{H}^+$ называется трансляционно инвариантным, если
$$
\forall(t \geq 0)\left[T(t)\mathcal{H}^+ \subset \mathcal{H}^+ \right]
$$

\paragraph{Элементарный пример} (\cite{Vorotnikov}, замечание 4.2.13, стр. 97).

Рассмотрим дифференциальное уравнение
\begin{equation}\label{primer_iz_statyi}
	u'(t)=
	\left\{
		\begin{array}{ll}
			-(u(t)-1)^2, & u(t) > 1, \\
			-u^2 (t)   , & 0 \leq u(t) \leq 1 \\
			u^2 (t)    , & u(t) < 0
		\end{array}
	\right.
\end{equation}

Заметим, что такая форма записи может поначалу смутить неподготовленного читателя, особенно --- студента (по крайней мере, меня смутила - прим. авт.).
Как правило, привычным является нечто вроде
\begin{equation*}
	sgn(t)=
	\left\{
		\begin{array}{ll}
			-1, & t < 0, \\
			 0, & t = 0 \\
			 1, & t > 0
		\end{array}
	\right.
\end{equation*}
Здесь же в правой части выражение зависит от $u$.
Тем не менее, это совершенно правомерно.
В общем виде дифференциальное уравнение выглядит как $u'(t)=f(u,t)$,
поэтому запись (\ref{primer_iz_statyi}) корректна;
с точки зрения физического смысла зависимость производной функции $u$ от значения этой функции логична:
такое условие как будто бы <<смотрит>>, куда попала функция, и в зависимости от этого <<выдаёт>> ей приращение.


Будем искать такие решения, что
\begin{equation}
	u(t) \in C[0; +\infty)
\end{equation}


...

Рассмотрим обыкновенное дифференциальное уравнение
\begin{equation}\label{primer_p1}
	u'(t)=-(u(t)-1)^2
\end{equation}

Решим его как уравнение с разделяющимися переменными:
$$
	\frac{du}{dt}=-(u-1)^2
$$

Заметим, что $u\equiv 1$ --- решение уравнения (\ref{primer_p1}).
Умножив обе части на $\frac{dt}{(u-1)^2}$, имеем:
$$
	\frac{du}{-(u-1)^2}=dt
$$

Проинтегрируем обе части:
$$
	\frac{du}{u-1}=t+C
$$

Выразим $u$:
\begin{equation}\label{primer_p1_solution}
	u=\frac{1}{t+C}+1
\end{equation}
где $C\in\R$.

Заметим, что условиям $u(t) \in C[0; +\infty)$ и $u(t)>1$ удовлетворяют только решения вида \ref{primer_p1_solution}, где $С>0$:
\begin{equation}
	u=\frac{1}{t+C}+1, C>0
\end{equation}


\addcontentsline{toc}{chapter}{Литература}
\begin{thebibliography}{99}

\bibitem{Kondratyev} Аттракторы уравнений неньютоновской гидродинамики / В. Г. Звягин, С. К. Кондратьев. – Успехи математических наук, 2014, сентябрь-октябрь, т. 69, вып 5 (419). – 76с.

\bibitem{Vorotnikov} Topological Approximation Methods for Evolutionary Problems of Nonlinear Hydrodynamics / Victor G. Zvyagin, Dmitry A. Vorotnikov. Walter de Gruyter, Berlin, New York. 2008.

\bibitem{Fursikov} Оптимальное управление распределёнными системами. Теория и приложения / А. В. Фурсиков. Новосибирск. Научная книга, 1999.

\bibitem{Zelenaya} Аттракторы для уравнений моделей движения вязкоупругих сред : учебное пособие / В.\,Г.\,Звягин, С.\,К.\,Кондратьев~; Воронежский государственный университет. -- Воронеж : Издательско-полиграфический центр Воронежского государственного университета, 2010. -- 266 с.


\end{thebibliography}

\end{document}
