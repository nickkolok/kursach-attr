\input{lib/packs}
\LARGE
\begin{document}
\input{lib/macro}

Курсовая работа


Непонятно: где брать образец титульного листа.


\opred
Траекторией (решением) дифференциального уравнения называется функция, при подстановке обращающая его в тождественное равенство на треубемом множестве значений аргумента.


Непонятно: чем может быть задано пространство траекторий, кроме как уравнением?
Полугруппой?
Ещё чем-то?


\opred
Пространством траекторий заданного дифференциального уравнений называется множество его решений.

Заметим, что пространство траекторий, хотя и называется пространством, может не иметь структуры линейного пространства:
сумма (в поточечном смысле) решений дифференициального уравнения вовсе не обязательно является его решением.

Рассмотрим, например, обыкновенное дифференциальное уравнение $y'(t) = 2t$.
Функции $f(t)=t^2$ и $g(t)=t^2 +1$ являются его решениями, в то время как их сумма $h(t) = f(t) + g(t) = 2t^2 + 1$ не является решением.


Необходимо: выяснить, подпространством какого пространства является пространство траекторий.
Предположительно $C_{[0;+\infty)}$.


Необходимо: определение: аттрактор.


Необходимо: определение: траекторный аттрактор.


Необходимо: определение: минимальный траекторный аттрактор.

В \cite{Kondratyev} (сразу после определения 4.5) указывается на смену терминологии в статье.
Правильно ли я понимаю, что следует пользоваться терминологией из 4 части?


Необходимо: определение: полугруппа.


Необходимо: определение: сдвиг.




\addcontentsline{toc}{chapter}{Литература}
\begin{thebibliography}{99}

\bibitem{Kondratyev} Аттракторы уравнений неньютоновской гидродинамики / В. Г. Звягин, С. К. Кондратьев. – Успехи математических наук, 2014, сентябрь-октябрь, т. 69, вып 5 (419). – 76с.

\bibitem{Vorotnikov} Topological Approximation Methods for Evolutionary Problems of Nonlinear Hydrodynamics / Victor G. Zvyagin, Dmitry A. Vorotnikov. Walter de Gruyter, Berlin, New York. 2008.

\bibitem{Fursikov} Topological Оптимальное управление распределёнными системами. Теория и приложения / А. В. Фурсиков. Новосибирск. Научная книга, 1999.

\end{thebibliography}

\end{document}
