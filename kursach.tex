\documentclass[a4paper,14pt]{report} %размер бумаги устанавливаем А4, шрифт 12пунктов
\usepackage[T2A]{fontenc}
\usepackage[utf8]{inputenc}
\usepackage[english,russian]{babel} %используем русский и английский языки с переносами
\usepackage{amssymb,amsfonts,amsmath,mathtext,cite,enumerate,float,amsthm} %подключаем нужные пакеты расширений
\usepackage[pdftex,unicode,colorlinks=true,linkcolor=blue]{hyperref}
\usepackage{indentfirst} % включить отступ у первого абзаца
\usepackage[dvips]{graphicx} %хотим вставлять рисунки?
\graphicspath{{illustr/}}%путь к рисункам

\makeatletter
\renewcommand{\@biblabel}[1]{#1.} % Заменяем библиографию с квадратных скобок на точку:
\makeatother %Смысл этих трёх строчек мне непонятен, но поверим "Запискам дебианщика"

\usepackage{geometry} % Меняем поля страницы. 
\geometry{left=1cm}% левое поле
\geometry{right=1cm}% правое поле
\geometry{top=1cm}% верхнее поле
\geometry{bottom=2cm}% нижнее поле

\renewcommand{\theenumi}{\arabic{enumi}}% Меняем везде перечисления на цифра.цифра
\renewcommand{\labelenumi}{\arabic{enumi}}% Меняем везде перечисления на цифра.цифра
\renewcommand{\theenumii}{.\arabic{enumii}}% Меняем везде перечисления на цифра.цифра
\renewcommand{\labelenumii}{\arabic{enumi}.\arabic{enumii}.}% Меняем везде перечисления на цифра.цифра
\renewcommand{\theenumiii}{.\arabic{enumiii}}% Меняем везде перечисления на цифра.цифра
\renewcommand{\labelenumiii}{\arabic{enumi}.\arabic{enumii}.\arabic{enumiii}.}% Меняем везде перечисления на цифра.цифра


\LARGE
\begin{document}
\newcommand{\pp}{Предположим противное}
%\newcommand{\pp}{{\LARП\!\!\!\!п~}}
\newcommand{\dokvo}{\paragraph{Доказательство.}}
\newcommand{\dokno}{\textbf {Доказано.}}
\newcommand{\neobh}{\paragraph{Необходимость.}}
\newcommand{\dost }{\paragraph{Достаточность.}}
\newcommand{\opred}{\paragraph{Определение.}}
\newcommand{\mnemo}{\paragraph{Мнемоника.}}
\newcommand{\N}{\mathbb{N}}
\newcommand{\Z}{\mathbb{Z}}
\newcommand{\Q}{\mathbb{Q}}
\newcommand{\R}{\mathbb{R}}
\newcommand{\one}[1]{\mathbb{I}_{#1}}
\renewcommand{\C}{\mathbb{C}}
\newcommand{\Beta}{B}%Костыль, а что поделать?
\newcommand{\Rn}{$\mathbb{R}^n~$}
\newcommand{\Rm}{$\mathbb{R}^m~$}
\renewcommand{\epsilon}{\varepsilon}
\renewcommand{\geq}{\geqslant}
\renewcommand{\leq}{\leqslant}
\newcommand{\fXR}{Пусть $X \subset \R, f:X \to \R$ }
\newcommand{\fXRx}{\fXR, $x_0$ - предельная точка $X$ }
\newcommand{\sgn}{\mathrm{sgn}~}
\newcommand{\nid}{\Leftrightarrow}
\newcommand{\intl}{\int\limits}
\newcommand{\suml}{\sum\limits}
\newcommand{\Models}{|\!\!\!=\!\!\!|}
\newcommand{\Rightleftarrow}{\Leftrightarrow}

\newcommand{\xI}{{\vec{\xi}}}%Костыль для тервера, очень уж там часто встречается
\newcommand{\calF}{\mathcal{F}}
\newcommand{\calB}{\mathcal{B}}
\newcommand{\GOFP}{$G \sim \left<\Omega,\calF,P\right>$}

\newenvironment{zamena}[1][c]{=\left<\begin{array}{#1}}{\end{array}\right>=}

\newtheorem{theorem}{Теорема}[section]
\newenvironment{teorema}[1][{}]{\begin{theorem}{#1}\upshape}{\end{theorem}}

\theoremstyle{definition}

\newtheorem{zamech}{Замечание}[section]
\newtheorem{primer}{Пример}[section]
\newtheorem{opr}{Определение}[section]

\newtheorem{sledstvie}{Следствие}[theorem]
\newtheorem{utverzhd}[theorem]{Утверждение}
\newtheorem{lemma}[theorem]{Лемма}

\long\def\comment{}


Курсовая работа


Непонятно: где брать образец титульного листа.


\opred
Решением дифференциального уравнения называется функция, при подстановке обращающая его в тождественное равенство на требуемом множестве значений аргумента.

\opred
Пусть $x(t)$ --- решение дифференциального уравнения.
Тогда график этого решения, т. е. множество $\{(t,x(t))\}$,
где $t$ пробегает промежуток, на котором данное дифференциальное уравнение решается,
называется траекторией дифференциального уравнения.


Непонятно: чем может быть задано пространство траекторий, кроме как уравнением?
Полугруппой?
Ещё чем-то?


\opred
Пространством траекторий заданного дифференциального уравнений называется множество его траекторий.

Заметим, что пространство траекторий, хотя и называется пространством, может не иметь структуры линейного пространства:
сумма (в поточечном смысле) решений дифференициального уравнения вовсе не обязательно является его решением,
соответственно, график суммы решений может не быть траекторией.

Рассмотрим, например, обыкновенное дифференциальное уравнение $y'(t) = 2t$.
Функции $f(t)=t^2$ и $g(t)=t^2 +1$ являются его решениями, в то время как их сумма $h(t) = f(t) + g(t) = 2t^2 + 1$ не является решением.


Необходимо: выяснить, подмножеством какого пространства является пространство траекторий.
Предположительно $C_{[0;+\infty)}$.





Необходимо: определение: аттрактор.


Необходимо: определение: траекторный аттрактор.


Необходимо: определение: минимальный траекторный аттрактор.

В \cite{Kondratyev} (сразу после определения 4.5) указывается на смену терминологии в статье.
Правильно ли я понимаю, что следует пользоваться терминологией из 4 части?


Необходимо: определение: полугруппа.


\opred (\cite{Zelenaya}, стр. 118)
Оператором сдвига $T(h)$, $h\in\R$ называется оператор, который функции $f$ ставит в соответствие функцию $T(h)f$, такую, что
$$
T(h)f(t)=f(t+h)
$$


\opred (\cite{Zelenaya}, стр. 121)
Пространство траекторий $\mathcal{H}^+$ называется трансляционно инвариантным, если
$$
\forall(t \geq 0)\left[T(t)\mathcal{H}^+ \subset \mathcal{H}^+ \right]
$$


В теории аттракторов изучаются дифференциальные уравнения (как обыкновенные дифференциальные уравнения, так и уравнения в частных производных) такие, что их решения $x(t)$ определены на $\mathbb{R}_+$ и принимают значения в некотором рефлексивном банаховом пространстве $E$.

Пусть $E_0$ --- также банахово пространство, и притом вложение $E \subset E_0$ непрерывно.
Будем искать такие решения изучаемого дифференциального уравнения, которые принадлежат пространству $C(\mathbb{R},E_0) \cap L_\infty(\mathbb{R},E)$.
Стоит упомянуть, что норма в пространстве $E$ не обязательно индуцирована нормой в пространстве $E_0$.

Заметим вскользь, что в природе никаких банаховых пространств нет; все эти абстракции вводятся математикой исключительно из соображений целесообразности решения уравнения.

\paragraph{Обозначение.}
В дальнейшем обозначим $T_+ = C(\mathbb{R},E_0) \cap L_\infty(\mathbb{R},E)$

\opred
Фазовым пространством дифференицального уравнения называют множество состояний системы, которую оно описывает.

В теории аттракторов в качестве фазового пространства принимается рефлексивное банахово пространство $E$.


\paragraph{Элементарный пример} (\cite{Vorotnikov}, замечание 4.2.13, стр. 97).

Рассмотрим дифференциальное уравнение
\begin{equation}\label{primer_iz_statyi}
	u'(t)=
	\left\{
		\begin{array}{ll}
			-(u(t)-1)^2, & u(t) > 1, \\
			-u^2 (t)   , & 0 \leq u(t) \leq 1 \\
			u^2 (t)    , & u(t) < 0
		\end{array}
	\right.
\end{equation}

Заметим, что такая форма записи может поначалу смутить неподготовленного читателя, особенно --- студента (по крайней мере, меня смутила - прим. авт.).
Как правило, привычным является нечто вроде
\begin{equation*}
	sgn(t)=
	\left\{
		\begin{array}{ll}
			-1, & t < 0, \\
			 0, & t = 0 \\
			 1, & t > 0
		\end{array}
	\right.
\end{equation*}
Здесь же в правой части выражение зависит от $u$.
Тем не менее, это совершенно правомерно.
В общем виде дифференциальное уравнение выглядит как $u'(t)=f(u,t)$,
поэтому запись (\ref{primer_iz_statyi}) корректна;
с точки зрения физического смысла зависимость производной функции $u$ от значения этой функции логична:
такое условие как будто бы <<смотрит>>, куда попала функция, и в зависимости от этого <<выдаёт>> ей приращение.


Будем искать такие решения, что
\begin{equation}
	u(t) \in C[0; +\infty)
\end{equation}

Решим сначала каждое из трёх дифференциальных уравнений в отдельности, а затем отберём решения,
удовлетворяющие соответствующим условиям.

\begin{enumerate}

\item)
Рассмотрим обыкновенное дифференциальное уравнение
\begin{equation}\label{primer_p1}
	u'(t)=-(u(t)-1)^2
\end{equation}

Решим его как уравнение с разделяющимися переменными:
$$
	\frac{du}{dt}=-(u-1)^2
$$

Заметим, что $u\equiv 1$ --- решение уравнения (\ref{primer_p1}).
Умножив обе части на $\frac{dt}{(u-1)^2}$, имеем:
$$
	\frac{du}{-(u-1)^2}=dt
$$

Проинтегрируем обе части:
$$
	\frac{1}{u-1}=t+C
$$

Выразим $u$:
\begin{equation}\label{primer_p1_solution}
	u=\frac{1}{t+C}+1
\end{equation}
где $C\in\R$.

Заметим, что условиям $u(t) \in C[0; +\infty)$ и $u(t)>1$ удовлетворяют только решения вида (\ref{primer_p1_solution}), где $C>0$:
\begin{equation}
	u=\frac{1}{t+C}+1, C>0
\end{equation}


\item)

Рассмотрим обыкновенное дифференциальное уравнение
\begin{equation}\label{primer_p2}
	u'(t)=-u^2(t)
\end{equation}

Решим его как уравнение с разделяющимися переменными:
$$
	\frac{du}{dt}=-u^2
$$

Заметим, что $u\equiv 0$ --- решение уравнения (\ref{primer_p2}).
Умножив обе части на $\frac{dt}{u^2}$, имеем:
$$
	\frac{du}{-u^2}=dt
$$

Проинтегрируем обе части:
$$
	\frac{1}{u}=t+1+C
$$

Выразим $u$:
\begin{equation}\label{primer_p2_solution}
	u=\frac{1}{t+1+C}
\end{equation}
где $C\in\R$.

Заметим, что условиям $u(t) \in C[0; +\infty)$ и $0 \leq u(t) \leq 1$ удовлетворяют только решения вида (\ref{primer_p2_solution}), где $C>0$:
\begin{equation}
	u=\frac{1}{t+1+C}, C>0
\end{equation}
и решение $u \equiv 0$.




\item)

Рассмотрим обыкновенное дифференциальное уравнение
\begin{equation}\label{primer_p3}
	u'(t)=-u^2(t)
\end{equation}

Решим его как уравнение с разделяющимися переменными:
$$
	\frac{du}{dt}=u^2
$$

Заметим, что $u\equiv 0$ --- решение уравнения (\ref{primer_p3}).
Умножив обе части на $\frac{dt}{u^2}$, имеем:
$$
	\frac{du}{u^2}=dt
$$

Проинтегрируем обе части:
$$
	\frac{1}{u}=-(t+C)
$$

Выразим $u$:
\begin{equation}\label{primer_p3_solution}
	u=\frac{1}{-(t+C)}
\end{equation}
где $C\in\R$.

Заметим, что условиям $u(t) \in C[0; +\infty)$ и $u(t) < 0$ удовлетворяют только решения вида (\ref{primer_p3_solution}), где $C>0$:
\begin{equation}
	u=-\frac{1}{t+C}, C>0.
\end{equation}
Решение $u \equiv 0$ не удовлетворяет условию $u(t)<0$.

\end{enumerate}

Таким образом, решения уравнения (\ref{primer_iz_statyi}) выписываются в виде:
\begin{equation}\label{primer_iz_statyi_u_t}
	\left[
		\begin{array}{l}
			u=\frac{1}{t+C}+1
		\\\\
			u=\frac{1}{t+1+C}
		\\\\
			u=0
		\\\\
			u=-\frac{1}{t+C},
		\end{array}
	\right.
\end{equation}
где $C>0$ (во втором случае $C \geq 0$).

Покажем теперь разрешимость порождаемой уравнением (\ref{primer_iz_statyi}) задачи Коши.
В силу того, что уравнение (\ref{primer_iz_statyi}) автономно, т.~е. его правая часть зависит только от $u$ и не зависит от $t$,
достаточно показать разрешимость задачи Коши с начальным условием в нуле.

Итак, пусть $u(0) = p$.
Рассмотрим четыре случая.

\begin{enumerate}

\item)
Пусть $p>1$.
Тогда по условию выпущенное из $p$ решение должно удовлетворять уравнению (\ref{primer_p1}).
Подставив $t=0$ в (\ref{primer_p1_solution}), получаем следующее уравнение для поиска $C$:
$$
	p=\frac{1}{0+C}+1
$$
Решим его:
$$
	p=\frac{1}{C}+1
$$
$$
	p-1=\frac{1}{C}
$$
$$
	C=\frac{1}{p-1}
$$
Заметим, что при $p>1$ имеем константу $C>0$, определяемую единственным образом.
Значит, решение задачи Коши при $p>1$ единственно и имеет вид:
\begin{equation}\label{primer_zk_1_0}
	u_1(t)=\frac{1}{t+\frac{1}{p-1}}+1
\end{equation}

Или, что то же самое,
\begin{equation}\label{primer_zk_1}
	u_1(t)=\frac{p+pt-t}{pt-t+1}
\end{equation}

Заметим, что из представления (\ref{primer_zk_1_0}) следует, что
$$
	\forall(p>1)\forall\left(t \in \mathbb{R}_+\right)\left[u_1(t) > 1\right],
$$
а значит, формула (\ref{primer_zk_1}) задаёт решение уравнения (\ref{primer_p1}) на всей неотрицательной полуоси $\mathbb{R}_+$.

\item)
Пусть $0<p \leq 1$.
Тогда по условию выпущенное из $p$ решение должно удовлетворять уравнению (\ref{primer_p2}).
Подставив $t=0$ в (\ref{primer_p2_solution}), получаем следующее уравнение для поиска $C$:
$$
	p=\frac{1}{0+1+C}
$$
Решим его:
$$
	C=\frac{1}{p}-1
$$
Заметим, что при $0<p \leq 1$ имеем константу $C \geq 0$, определяемую единственным образом.
Значит, решение задачи Коши при $0<p<1$ единственно и имеет вид:
\begin{equation}\label{primer_zk_2_0}
	u_2(t)=\frac{1}{t+\frac{1}{p}}
\end{equation}

Или, что то же самое,
\begin{equation}\label{primer_zk_2}
	u_2(t)=\frac{p}{pt+1}
\end{equation}

Это решение снова существует на всей неотрицательной числовой полуоси $\mathbb{R}_+$, так как
$$
	\forall(0<p \leq 1)\forall\left(t \in \mathbb{R}_+\right)\left[0 < u_2(t) < 1\right].
$$

\item)
Положим теперь $p=0$.
Тогда по условию выпущенное из $p$ решение должно удовлетворять уравнению (\ref{primer_p2}).
Искомое решение есть тождественный нуль:
\begin{equation}\label{primer_zk_3}
	u_3 \equiv 0
\end{equation}
Покажем, что других решений задачи Коши нет.
Предположим противное.
Тогда решение задаётся формулой (\ref{primer_p2_solution}) и отвечает условию
$$
	0=u_{3^{'}}(0)=\frac{1}{t+1+C},
$$
правая часть которого не обращается в ноль ни при каком $C$.
Получили противоречие.
Следовательно, при $p=0$ задача Коши для исследуемого уравнения также имеет единственное решение на всей числовой полуоси $\mathbb{R}_+$.

\item)
Пусть наконец $p<0$.
Тогда по условию выпущенное из $p$ решение должно удовлетворять уравнению (\ref{primer_p3}).
Подставив $t=0$ в (\ref{primer_p3_solution}), получаем следующее уравнение для поиска $C$:
$$
	p=\frac{1}{-(0+C)}
$$
Решим его:
$$
	C=-\frac{1}{p}
$$
Заметим, что при $p<0$ имеем константу $C>0$, определяемую единственным образом.
Значит, решение задачи Коши при $p<0$ единственно и имеет вид:
\begin{equation}\label{primer_zk_4_0}
	u_4(t)=-\frac{1}{t-\frac{1}{p}}
\end{equation}

Или, что то же самое,
\begin{equation}\label{primer_zk_4}
	u_4(t)=\frac{p}{1-pt}
\end{equation}

Это решение снова существует на всей неотрицательной числовой полуоси $\mathbb{R}_+$, так как
$$
	\forall(p<0)\forall\left(t \in \mathbb{R}_+\right)\left[u_4(t) < 0\right].
$$





\end{enumerate}

\opred
Оператором сдвига $S^{t_0}_t (p)$ по траекториям дифференциального уравнения $x'(t) = f(u,t)$ называется функция, такая, что
\begin{equation*}
	S^{t_0}_t (p) = q \Rightleftarrow
		\exists\left(x_p(t) : x_p(t_0) = p\right)\left[x_p(t) = q\right].
\end{equation*}

Т.о. оператор сдвига сдвигает точку по траектории дифференциального уравнения с момента времени $t_0$ до момента времени $t$.

\paragraph{Обозначение.}
В случае, если $t_0=0$, соответствующий оператор сдвига по траекториям дифференциального уравнения $S^{t_0}_t (p)$ в целях упрощения записи  в дальнейшем будем обозначать просто $S_t (p)$.

Выпишем теперь оператор сдвига (в нуле) по траекториям уравнения (\ref{primer_iz_statyi}).
Из (\ref{primer_zk_1}) следует, что при $p>1$
\begin{equation}\label{sdvig_p1}
	S_t p =\frac{p+pt-t}{pt-t+1}
\end{equation}

Из (\ref{primer_zk_2}) следует, что при $0 < p \leq 1$
\begin{equation}\label{sdvig_p2}
	S_t p =\frac{p}{pt+1}
\end{equation}

Из (\ref{primer_zk_3}) следует, что при $p = 0$
\begin{equation}\label{sdvig_p3}
	S_t p =\frac{p}{pt+1}
\end{equation}

Из (\ref{primer_zk_4}) следует, что при $p < 0$
\begin{equation}\label{sdvig_p4}
	S_t p =\frac{p}{1 - pt}
\end{equation}

Или, объединяя (\ref{sdvig_p1})--(\ref{sdvig_p4}),
\begin{equation}
	S_t p =
	\left\{
		\begin{array}{ll}
			\frac{p+pt-t}{pt-t+1}, & p > 1
		\\\\
			\frac{p}{pt+1},        & 0 \leq p \leq 1
		\\\\
			\frac{p}{1 - pt},      & p < 0
		\end{array}
	\right.
\end{equation}

Перед переходом непосредственно к рассмотрению аттракторов введём ещё два вспомогательных определения.

\opred (\cite{Zelenaya}, параграф 3.2, стр. 121)

Пусть $B \subset T_+$.
Сечением множества траекторий $B$ в момент времени $t \geq 0$ называется множество
$$
	B(t)=\left\{u(t) : u \in B \right\} \subset E.
$$


\opred (\cite{Zelenaya}, параграф 3.2, стр. 122, из опр. 3.2.1)

Пусть $R,Q \subset E$.
Полуотклонением в пространстве $E_0$ множества $R$ от множества $Q$ называется величина
$$
	h_{E_0}(R,Q) = \sup_{r\in R} \inf_{q \in Q} \| q - r \|_{E_0},
$$
т.е.
$$
	h_{E_0} : 2^E \times 2^E \to \mathbb{R},
$$
где $2^E$ --- множество всевозможных подмножеств пространства $E$.

Заметим, что полутклонение --- не симметричная операция, как могло показаться торопливому читателю (мне сначала показалось --- прим. авт.).
Приведём пример случая, когда
$$
	h_{E_0}(R,Q) \neq h_{E_0}(Q,R).
$$

Пусть $E_0 = E = \mathbb{R}$, $\|x\|_{\mathbb{R}} = |x|$,
$$
	R =\{1\},
$$
$$
	Q=\{0,3\}.
$$
Тогда
$$
	h_{E_0}(R,Q) =
	\sup_{r\in R} \inf_{q \in Q} \| q - r \|_{E_0} =
	\sup_{r\in \{1\}} \inf_{q \in \{0,3\}} | q - r | =
	\inf_{q \in \{0,3\}} | q - 1 | =
	\min\{|0-1|,|3-1|\} =
	\min\{1,2\} =
	1;
$$
с другой стороны,
$$
	h_{E_0}(Q,R) =
	\sup_{q \in Q} \inf_{r\in R} \| r - q \|_{E_0} =
	\sup_{q \in \{0,3\}} \inf_{r\in \{1\}} | r - q | =
	\sup_{q \in \{0,3\}} | 1 - q | =
	\max\{|1-0|,|1-3|\} =
	\max\{1,2\} =
	2.
$$
Т.е.
$$
	h_{\mathbb{R}}\left( \{1\}, \{0,3\}\right) \neq h_{\mathbb{R}}\left( \{0,3\} , \{1\} \right).
$$

Идеологический смысл несимметричности полуотклонения заключается в том,
что первое множество считается <<плохим>>, а второе --- <<хорошим>>;
полуотклонение показывает, насколько <<худшему>> представителю <<плохого>> множества далеко до ближайшего представителя <<хорошего>> множества.

Перейдём теперь к центральному понятию, рассматриваемому в данной работе --- понятию аттрактора.

Название <<аттрактор>> (англ. <<attractor>>) буквально означает <<притягивающий>>.
Аттрактором называют множество, которое на бесконечности <<притягивает>> решения изучаемого уравнения.
В зависимости от того, какой смысл вкладывается в понятие <<притягивать>>,
выделяют несколько видов аттракторов.

\opred (\cite{Zelenaya}, параграф 3.2, стр. 121-122, опр. 3.2.1)

Множество  $\mathcal{A} \subset E $ называется глобальным аттрактором (в $E_0$) для пространства траекторий $\mathcal{H}^+$, если:
\begin{enumerate}
	\item)
		$\mathcal{A}$ компактно в $E_0$ и ограничено в $E$;
	\item)
		$
			\forall(K>0)\forall(B \subset \mathcal{H}^+ \cap B_{L_{\infty}\left( \mathbb{R}_+; E \right)}(0,K))
				\left[
					h_{E_0}(B(t),\mathcal{A}) \xrightarrow[t\to \infty]{}{0}
				\right],
		$
		т.е. сечения любого ограниченного в $L_{\infty}\left( \mathbb{R}_+; E \right)$ множества с течением времени <<притягиваются>> к $\mathcal{A}$.
	\item)
		$\mathcal{A}$ --- наименьшее по включению множество, удовлетворяющее условиям (1) и (2).
\end{enumerate}


Глобальный аттрактор, как следует из определения, принадлежит фазовому пространству;
как помнит внимательный читатель, фазовое пространство есть множество всевозможных состояний системы, которая моделируется изучаемым уравнением.
Глобальный аттрактор представляет собой набор состояний, к которым система будет стремиться со временем;
все остальные состояния системы, т.е. точки фазового пространства, не входящие в глобальный аттрактор, преходящи;
с учётом того, что в реальных измерениях всегда присутствует некоторая погрешность, тем фактом, что решения могут не достигать глобального аттрактора, а только приближаться к нему, можно пренебречь и изучать только состояния системы, соответствующие глобальному аттрактору.


\opred (\cite{Zelenaya}, параграф 3.1, стр. 120)

Пусть $f:\R\to E$.
Тогда
$$
	\Pi_+f = \Bigl. f \Bigr|_{[0;+\infty)}
$$
и
$$
	\Pi_+f:[0; +\infty) \to E
$$

Смысл этого обозначения состоит в указании на тот факт, что при работе с функцией должна использоваться норма соответствующего пространства.
Кроме того, с физической точки зрения логично <<смотреть вперёд>>, т.~е. изучать поведение некоторых систем, только начиная с определённого момента времени, который и <<назначается>> нулевым.

\addcontentsline{toc}{chapter}{Литература}
\begin{thebibliography}{99}

\bibitem{Kondratyev} Аттракторы уравнений неньютоновской гидродинамики / В. Г. Звягин, С. К. Кондратьев. – Успехи математических наук, 2014, сентябрь-октябрь, т. 69, вып 5 (419). – 76с.

\bibitem{Vorotnikov} Topological Approximation Methods for Evolutionary Problems of Nonlinear Hydrodynamics / Victor G. Zvyagin, Dmitry A. Vorotnikov. Walter de Gruyter, Berlin, New York. 2008.

\bibitem{Fursikov} Оптимальное управление распределёнными системами. Теория и приложения / А. В. Фурсиков. Новосибирск. Научная книга, 1999.

\bibitem{Zelenaya} Аттракторы для уравнений моделей движения вязкоупругих сред : учебное пособие / В.\,Г.\,Звягин, С.\,К.\,Кондратьев~; Воронежский государственный университет. -- Воронеж : Издательско-полиграфический центр Воронежского государственного университета, 2010. --- 266 с.

\bibitem{zhidkosti_s_pamyatyu} Аттракторы слабых решений регуляризованной системы уравнений движения жидких сред с памятью / В.\,Г.\,Звягин, С.\,К.\,Кондратьев~; Известия вузов: Математика, 2011, № 8 --- c. 86–89

\end{thebibliography}

\end{document}
