\documentclass[a4paper,14pt]{report} %размер бумаги устанавливаем А4, шрифт 12пунктов
\usepackage[T2A]{fontenc}
\usepackage[utf8]{inputenc}
\usepackage[english,russian]{babel} %используем русский и английский языки с переносами
\usepackage{amssymb,amsfonts,amsmath,mathtext,cite,enumerate,float,amsthm} %подключаем нужные пакеты расширений
\usepackage[pdftex,unicode,colorlinks=true,linkcolor=blue]{hyperref}
\usepackage{indentfirst} % включить отступ у первого абзаца
\usepackage[dvips]{graphicx} %хотим вставлять рисунки?
\graphicspath{{illustr/}}%путь к рисункам

\makeatletter
\renewcommand{\@biblabel}[1]{#1.} % Заменяем библиографию с квадратных скобок на точку:
\makeatother %Смысл этих трёх строчек мне непонятен, но поверим "Запискам дебианщика"

\usepackage{geometry} % Меняем поля страницы. 
\geometry{left=1cm}% левое поле
\geometry{right=1cm}% правое поле
\geometry{top=1cm}% верхнее поле
\geometry{bottom=2cm}% нижнее поле

\renewcommand{\theenumi}{\arabic{enumi}}% Меняем везде перечисления на цифра.цифра
\renewcommand{\labelenumi}{\arabic{enumi}}% Меняем везде перечисления на цифра.цифра
\renewcommand{\theenumii}{.\arabic{enumii}}% Меняем везде перечисления на цифра.цифра
\renewcommand{\labelenumii}{\arabic{enumi}.\arabic{enumii}.}% Меняем везде перечисления на цифра.цифра
\renewcommand{\theenumiii}{.\arabic{enumiii}}% Меняем везде перечисления на цифра.цифра
\renewcommand{\labelenumiii}{\arabic{enumi}.\arabic{enumii}.\arabic{enumiii}.}% Меняем везде перечисления на цифра.цифра


\LARGE
\begin{document}
\newcommand{\pp}{Предположим противное}
%\newcommand{\pp}{{\LARП\!\!\!\!п~}}
\newcommand{\dokvo}{\paragraph{Доказательство.}}
\newcommand{\dokno}{\textbf {Доказано.}}
\newcommand{\neobh}{\paragraph{Необходимость.}}
\newcommand{\dost }{\paragraph{Достаточность.}}
\newcommand{\opred}{\paragraph{Определение.}}
\newcommand{\mnemo}{\paragraph{Мнемоника.}}
\newcommand{\N}{\mathbb{N}}
\newcommand{\Z}{\mathbb{Z}}
\newcommand{\Q}{\mathbb{Q}}
\newcommand{\R}{\mathbb{R}}
\newcommand{\one}[1]{\mathbb{I}_{#1}}
\renewcommand{\C}{\mathbb{C}}
\newcommand{\Beta}{B}%Костыль, а что поделать?
\newcommand{\Rn}{$\mathbb{R}^n~$}
\newcommand{\Rm}{$\mathbb{R}^m~$}
\renewcommand{\epsilon}{\varepsilon}
\renewcommand{\geq}{\geqslant}
\renewcommand{\leq}{\leqslant}
\newcommand{\fXR}{Пусть $X \subset \R, f:X \to \R$ }
\newcommand{\fXRx}{\fXR, $x_0$ - предельная точка $X$ }
\newcommand{\sgn}{\mathrm{sgn}~}
\newcommand{\nid}{\Leftrightarrow}
\newcommand{\intl}{\int\limits}
\newcommand{\suml}{\sum\limits}
\newcommand{\Models}{|\!\!\!=\!\!\!|}
\newcommand{\Rightleftarrow}{\Leftrightarrow}

\newcommand{\xI}{{\vec{\xi}}}%Костыль для тервера, очень уж там часто встречается
\newcommand{\calF}{\mathcal{F}}
\newcommand{\calB}{\mathcal{B}}
\newcommand{\GOFP}{$G \sim \left<\Omega,\calF,P\right>$}

\newenvironment{zamena}[1][c]{=\left<\begin{array}{#1}}{\end{array}\right>=}

\newtheorem{theorem}{Теорема}[section]
\newenvironment{teorema}[1][{}]{\begin{theorem}{#1}\upshape}{\end{theorem}}

\theoremstyle{definition}

\newtheorem{zamech}{Замечание}[section]
\newtheorem{primer}{Пример}[section]
\newtheorem{opr}{Определение}[section]

\newtheorem{sledstvie}{Следствие}[theorem]
\newtheorem{utverzhd}[theorem]{Утверждение}
\newtheorem{lemma}[theorem]{Лемма}

\long\def\comment{}



О некоторых примерах аттракторов дифференциальных уравнений.

                                   Н.Н.Авдеев

\paragraph{Введение.}
В настоящее время имеются несколько подходов к определению понятий аттракторов,
которые применяются для исследования дифференциальных уравнений.
Это траекторные и глобальные аттракторы инвариантных и неинвариантных траекторных пространств,
аттракторы полугруппы трансляций и т.д.
В настоящей статье подробно исследуются вопросы существования аттракторов
при разных подходах к их определению для двух примеров дифференциальных уравнений,
кратко описанных в книге \cite{Vorotnikov}.
Кроме того, эти примеры показывают, что при некоторых подходах аттракторы состоят не только из решений уравнений,
но включают и другие элементы.
Описание общих условий,
при которых элементы входят в аттракторы в теории траекторных и глобальных аттракторов неинвариантных траекторных пространств,
можно найти в \cite{Kondratyev}.


План работы.

Первый пример.
Далее приводите пример, потом приводите определение полугруппы трансляций и т.д. и показываете, что в этом подходе нет аттрактора.
Затем приводите определения  траекторного и глобального аттрактора неинвариантного траекторного пространства и показываете, что аттракторы существуют, причем не только из решений  и их сечений для дифференциального уравнения.
Далее приводите определения  траекторного и глобального аттрактора инвариантного траекторного пространства  и отмечаете, что в этом случае аттракторы состоят только из решений и их сечений  для дифференциального уравнения
(а в данном примере можно найти инвариантное траекторное пространство?).

И наконец проводите условия, при которых элементы принадлежат аттрактору неинвариантного траекторного пространства.


\section*{--- Черновик статьи ---}


\subsection*{Пример дифференциального уравнения}

Рассмотрим дифференциальное уравнение (\cite{Vorotnikov}, remark 4.2.13, p. 97)

\begin{equation}\label{primer_iz_statyi}
	u'(t)=
	\left\{
		\begin{array}{ll}
			-(u(t)-1)^2, & u(t) > 1, \\
			-u^2 (t)   , & 0 \leq u(t) \leq 1 \\
			u^2 (t)    , & u(t) < 0
		\end{array}
	\right.
\end{equation}

с точки зрения теории обыкновенных дифференциальных уравнений, выпишем его решения и операторы сдвига.

Будем искать такие решения, что
\begin{equation}
	u(t) \in C[0; +\infty)
\end{equation}

Решим сначала каждое из трёх дифференциальных уравнений в отдельности, а затем отберём решения,
удовлетворяющие соответствующим условиям.

\begin{enumerate}

\item)
Рассмотрим обыкновенное дифференциальное уравнение
\begin{equation}\label{primer_p1}
	u'(t)=-(u(t)-1)^2
\end{equation}

Решим его как уравнение с разделяющимися переменными:
$$
	\frac{du}{dt}=-(u-1)^2
$$

Заметим, что $u\equiv 1$ --- решение уравнения (\ref{primer_p1}).
Умножив обе части на $\frac{dt}{(u-1)^2}$, имеем:
$$
	\frac{du}{-(u-1)^2}=dt
$$

Проинтегрируем обе части:
$$
	\frac{1}{u-1}=t+C
$$

Выразим $u$:
\begin{equation}\label{primer_p1_solution}
	u=\frac{1}{t+C}+1
\end{equation}
где $C\in\R$.

Заметим, что условиям $u(t) \in C[0; +\infty)$ и $u(t)>1$ удовлетворяют только решения вида (\ref{primer_p1_solution}), где $C>0$:
\begin{equation}
	u=\frac{1}{t+C}+1, C>0
\end{equation}


\item)

Рассмотрим обыкновенное дифференциальное уравнение
\begin{equation}\label{primer_p2}
	u'(t)=-u^2(t)
\end{equation}

Решим его как уравнение с разделяющимися переменными:
$$
	\frac{du}{dt}=-u^2
$$

Заметим, что $u\equiv 0$ --- решение уравнения (\ref{primer_p2}).
Умножив обе части на $\frac{dt}{u^2}$, имеем:
$$
	\frac{du}{-u^2}=dt
$$

Проинтегрируем обе части:
$$
	\frac{1}{u}=t+1+C
$$

Выразим $u$:
\begin{equation}\label{primer_p2_solution}
	u=\frac{1}{t+1+C}
\end{equation}
где $C\in\R$.

Заметим, что условиям $u(t) \in C[0; +\infty)$ и $0 \leq u(t) \leq 1$ удовлетворяют только решения вида (\ref{primer_p2_solution}), где $C\geq0$:
\begin{equation}
	u=\frac{1}{t+1+C}, C\geq0
\end{equation}
и решение $u \equiv 0$.




\item)

Рассмотрим обыкновенное дифференциальное уравнение
\begin{equation}\label{primer_p3}
	u'(t)=-u^2(t)
\end{equation}

Решим его как уравнение с разделяющимися переменными:
$$
	\frac{du}{dt}=u^2
$$

Заметим, что $u\equiv 0$ --- решение уравнения (\ref{primer_p3}).
Умножив обе части на $\frac{dt}{u^2}$, имеем:
$$
	\frac{du}{u^2}=dt
$$

Проинтегрируем обе части:
$$
	\frac{1}{u}=-(t+C)
$$

Выразим $u$:
\begin{equation}\label{primer_p3_solution}
	u=\frac{1}{-(t+C)}
\end{equation}
где $C\in\R$.

Заметим, что условиям $u(t) \in C[0; +\infty)$ и $u(t) < 0$ удовлетворяют только решения вида (\ref{primer_p3_solution}), где $C>0$:
\begin{equation}
	u=-\frac{1}{t+C}, C>0.
\end{equation}
Решение $u \equiv 0$ не удовлетворяет условию $u(t)<0$.

\end{enumerate}

Таким образом, решения уравнения (\ref{primer_iz_statyi}) выписываются в виде:
\begin{equation}\label{primer_iz_statyi_u_t}
	\left[
		\begin{array}{l}
			u=\frac{1}{t+C}+1
		\\\\
			u=\frac{1}{t+1+C}
		\\\\
			u=0
		\\\\
			u=-\frac{1}{t+C},
		\end{array}
	\right.
\end{equation}
где $C>0$ (во втором случае $C \geq 0$).

Покажем теперь разрешимость порождаемой уравнением (\ref{primer_iz_statyi}) задачи Коши.
В силу того, что уравнение (\ref{primer_iz_statyi}) автономно, т.~е. его правая часть зависит только от $u$ и не зависит от $t$,
достаточно показать разрешимость задачи Коши с начальным условием в нуле.

Итак, пусть $u(0) = p$.
Рассмотрим четыре случая.

\begin{enumerate}

\item)
Пусть $p>1$.
Тогда по условию выпущенное из $p$ решение должно удовлетворять уравнению (\ref{primer_p1}).
Подставив $t=0$ в (\ref{primer_p1_solution}), получаем следующее уравнение для поиска $C$:
$$
	p=\frac{1}{0+C}+1
$$
Решим его:
$$
	p=\frac{1}{C}+1
$$
$$
	p-1=\frac{1}{C}
$$
$$
	C=\frac{1}{p-1}
$$
Заметим, что при $p>1$ имеем константу $C>0$, определяемую единственным образом.
Значит, решение задачи Коши при $p>1$ единственно и имеет вид:
\begin{equation}\label{primer_zk_1_0}
	u_1(t)=\frac{1}{t+\frac{1}{p-1}}+1
\end{equation}

Или, что то же самое,
\begin{equation}\label{primer_zk_1}
	u_1(t)=\frac{p+pt-t}{pt-t+1}
\end{equation}

Заметим, что из представления (\ref{primer_zk_1_0}) следует, что
$$
	\forall(p>1)\forall\left(t \in \mathbb{R}_+\right)\left[u_1(t) > 1\right],
$$
а значит, формула (\ref{primer_zk_1}) задаёт решение уравнения (\ref{primer_p1}) на всей неотрицательной полуоси $\mathbb{R}_+$.

\item)
Пусть $0<p \leq 1$.
Тогда по условию выпущенное из $p$ решение должно удовлетворять уравнению (\ref{primer_p2}).
Подставив $t=0$ в (\ref{primer_p2_solution}), получаем следующее уравнение для поиска $C$:
$$
	p=\frac{1}{0+1+C}
$$
Решим его:
$$
	C=\frac{1}{p}-1
$$
Заметим, что при $0<p \leq 1$ имеем константу $C \geq 0$, определяемую единственным образом.
Значит, решение задачи Коши при $0<p \leq 1$ единственно и имеет вид:
\begin{equation}\label{primer_zk_2_0}
	u_2(t)=\frac{1}{t+\frac{1}{p}}
\end{equation}

Или, что то же самое,
\begin{equation}\label{primer_zk_2}
	u_2(t)=\frac{p}{pt+1}
\end{equation}

Это решение снова существует на всей неотрицательной числовой полуоси $\mathbb{R}_+$, так как
$$
	\forall(0<p \leq 1)\forall\left(t \in \mathbb{R}_+\right)\left[0 < u_2(t) < 1\right].
$$

\item)
Положим теперь $p=0$.
Тогда по условию выпущенное из $p$ решение должно удовлетворять уравнению (\ref{primer_p2}).
Искомое решение есть тождественный нуль:
\begin{equation}\label{primer_zk_3}
	u_3 \equiv 0
\end{equation}
Покажем, что других решений задачи Коши нет.
Предположим противное.
Тогда решение задаётся формулой (\ref{primer_p2_solution}) и отвечает условию
$$
	0=u_{3^{'}}(0)=\frac{1}{t+1+C},
$$
правая часть которого не обращается в ноль ни при каком $C$.
Получили противоречие.
Следовательно, при $p=0$ задача Коши для исследуемого уравнения также имеет единственное решение на всей числовой полуоси $\mathbb{R}_+$.

\item)
Пусть наконец $p<0$.
Тогда по условию выпущенное из $p$ решение должно удовлетворять уравнению (\ref{primer_p3}).
Подставив $t=0$ в (\ref{primer_p3_solution}), получаем следующее уравнение для поиска $C$:
$$
	p=\frac{1}{-(0+C)}
$$
Решим его:
$$
	C=-\frac{1}{p}
$$
Заметим, что при $p<0$ имеем константу $C>0$, определяемую единственным образом.
Значит, решение задачи Коши при $p<0$ единственно и имеет вид:
\begin{equation}\label{primer_zk_4_0}
	u_4(t)=-\frac{1}{t-\frac{1}{p}}
\end{equation}

Или, что то же самое,
\begin{equation}\label{primer_zk_4}
	u_4(t)=\frac{p}{1-pt}
\end{equation}

Это решение снова существует на всей неотрицательной числовой полуоси $\mathbb{R}_+$, так как
$$
	\forall(p<0)\forall\left(t \in \mathbb{R}_+\right)\left[u_4(t) < 0\right].
$$

\end{enumerate}

\opred
Оператором сдвига $S^{t_0}_t (p)$ по траекториям дифференциального уравнения $x'(t) = f(u,t)$ называется функция, такая, что
\begin{equation*}
	S^{t_0}_t (p) = q \Rightleftarrow
		\exists\left(x_p(t) : x_p(t_0) = p\right)\left[x_p(t) = q\right].
\end{equation*}

Т.о. оператор сдвига сдвигает точку по траектории дифференциального уравнения с момента времени $t_0$ до момента времени $t$.

\paragraph{Обозначение.}
В случае, если $t_0=0$, соответствующий оператор сдвига по траекториям дифференциального уравнения $S^{t_0}_t (p)$ в целях упрощения записи  в дальнейшем будем обозначать просто $S_t (p)$.

Учитывая (\ref{primer_zk_1})--(\ref{primer_zk_4}), оператор сдвига (в нуле) по траекториям уравнения (\ref{primer_iz_statyi}) имеет вид:
\begin{equation}\label{oper_sdviga_primer_1}
	S_t p =
	\left\{
		\begin{array}{ll}
			\frac{p+pt-t}{pt-t+1}, & p > 1
		\\\\
			\frac{p}{pt+1},        & 0 \leq p \leq 1
		\\\\
			\frac{p}{1 - pt},      & p < 0
		\end{array}
	\right.
\end{equation}


\subsection*{Вспомогательные определения}

Дадим вспомогательные определения, которые потребуются при применении к примеру теории аттракторов.


\opred
Пространством траекторий называется непустое множество $\mathcal{H}^+$, такое, что
$$
	\mathcal{H}^+ \subset C_{[0;+\infty)} \cap L_\infty{[0;+\infty)}.
$$

\opred
Траекторией называется элемент пространства траекторий.

При рассмотрении дифференциального уравнения под пространством траекторий мы, как это часто делается, будем понимать множество решений этого уравнения.


\opred (\cite{Zelenaya}, стр. 118)
Оператором сдвига $T(h)$, $h\in\R$ называется оператор, который функции $f$ ставит в соответствие функцию $T(h)f$, такую, что
$$
T(h)f(t)=f(t+h)
$$


\opred (\cite{Zelenaya}, стр. 121)
Пространство траекторий $\mathcal{H}^+$ называется трансляционно инвариантным, если
$$
\forall(t \geq 0)\left[T(t)\mathcal{H}^+ \subset \mathcal{H}^+ \right]
$$


Пусть $E$ --- рефлексивное банахово пространство, $E_0$ --- также банахово пространство и вложение $E \subset E_0$ непрерывно.
Будем искать такие решения изучаемого дифференциального уравнения,
которые принадлежат пространству $C(\mathbb{R},E_0) \cap L_\infty(\mathbb{R},E)$.
Норма в пространстве $E$ не обязательно индуцирована нормой в пространстве $E_0$.

\paragraph{Обозначение.}
В дальнейшем обозначим $T_+ = C(\mathbb{R},E_0) \cap L_\infty(\mathbb{R},E)$

\opred
Фазовым пространством дифференицального уравнения называют множество состояний системы, которую оно описывает.

В теории аттракторов в качестве фазового пространства принимается рефлексивное банахово пространство $E$.



\opred (\cite{Zelenaya}, параграф 3.2, стр. 121)

Пусть $B \subset T_+$.
Сечением множества траекторий $B$ в момент времени $t \geq 0$ называется множество
$$
	B(t)=\left\{u(t) : u \in B \right\} \subset E.
$$


\opred (\cite{Zelenaya}, параграф 3.2, стр. 122, из опр. 3.2.1)

Пусть $R,Q \subset E$.
Полуотклонением в пространстве $E_0$ множества $R$ от множества $Q$ называется величина
$$
	h_{E_0}(R,Q) = \sup_{r\in R} \inf_{q \in Q} \| q - r \|_{E_0},
$$
т.е.
$$
	h_{E_0} : 2^E \times 2^E \to \mathbb{R},
$$
где $2^E$ --- множество всевозможных подмножеств пространства $E$.

Заметим, что полутклонение --- не симметричная операция.
Приведём пример случая, когда
$$
	h_{E_0}(R,Q) \neq h_{E_0}(Q,R).
$$

Пусть $E_0 = E = \mathbb{R}$, $\|x\|_{\mathbb{R}} = |x|$,
$$
	R =\{1\},
$$
$$
	Q=\{0,3\}.
$$
Тогда
$$
	h_{E_0}(R,Q) =
	\sup_{r\in R} \inf_{q \in Q} \| q - r \|_{E_0} =
	\sup_{r\in \{1\}} \inf_{q \in \{0,3\}} | q - r | =
	\inf_{q \in \{0,3\}} | q - 1 | =
	\min\{|0-1|,|3-1|\} =
	\min\{1,2\} =
	1;
$$
с другой стороны,
$$
	h_{E_0}(Q,R) =
	\sup_{q \in Q} \inf_{r\in R} \| r - q \|_{E_0} =
	\sup_{q \in \{0,3\}} \inf_{r\in \{1\}} | r - q | =
	\sup_{q \in \{0,3\}} | 1 - q | =
	\max\{|1-0|,|1-3|\} =
	\max\{1,2\} =
	2.
$$
Т.е.
$$
	h_{\mathbb{R}}\left( \{1\}, \{0,3\}\right) \neq h_{\mathbb{R}}\left( \{0,3\} , \{1\} \right).
$$

Идеологический смысл несимметричности полуотклонения заключается в том,
что первое множество считается <<плохим>>, а второе --- <<хорошим>>;
полуотклонение показывает, насколько <<худшему>> представителю <<плохого>> множества далеко до ближайшего представителя <<хорошего>> множества.


\subsection*{Аттрактор полугруппы трансляций}

Перейдём теперь к центральному понятию, рассматриваемому в данной работе --- понятию аттрактора.

Название <<аттрактор>> (англ. <<attractor>>) буквально означает <<притягивающий>>.
Аттрактором называют множество, которое на бесконечности <<притягивает>> решения изучаемого уравнения.
В зависимости от того, какой смысл вкладывается в понятие <<притягивать>>,
выделяют несколько видов аттракторов.



Введём теперь понятие аттрактора полугруппы, следуя [\cite{Vorotnikov}, \S 4.1.1],
и покажем, что в рассматриваемом примере аттрактора полугруппы трансляций нет.

\opred

Семейство отображений $S_t : E \to E$, $t \geq 0$ называется полугруппой трансляций,
если $S_0$ --- тождественное отображение и
$$
	\forall(t>0,\tau>0)[S_t \circ S_\tau = S_{t+\tau}]
$$

Здесь и далее под $E$ будем понимать банахово пространство.

\opred

Полугруппа $S_t$ называется ограниченной в $E$, если для любого ограниченного в $E$ множества $B \subset E$ множество $\bigcup\limits_{t\geq0}S_t B$ также ограничено в $E$.

Здесь и далее через $F$ будем обозначать топологическое пространство, такое, что $E \cap F \ne \varnothing$.

\opred

Множество $A$ называется инвариантным относительно полугруппы трансляций $S_t$, если
$$
	\forall(t\geq 0)[S_t A = A]
$$

\opred

Множество $P \subset F$ называется $(E,F)$-притягивающим для полугруппы трансляций $S_t$,
если для любого ограниченного множества $B \subset E$ и любой открытой окрестности $W$ множества $P$ в $F$ существует число $h\geq 0$ такое, что
$$
	\forall(t \geq h)[S_t B \subset W]
$$


\opred

Множество $A\subset E\cap F$ называется $(E,F)$-аттрактором полугруппы трансляций $S_t$, если

а) $A$ компактно в $F$ и ограничено в $E$;

б) $A$ инвариантно относительно полугруппы трансляций $S_t$.

в) $A$ является $(E,F)$-притягивающим для полугруппы трансляций $S_t$.


Возвращаясь к разбираемому примеру, попробуем применить только что данное определение к полугруппе трансляций (\ref{oper_sdviga_primer_1}).

(\ref{oper_sdviga_primer_1}) --- действительно полугруппа трансляций, т.к. явно задано $S_0 p = p$, а равенство $S_t \circ S_\tau = S_{t+\tau}$ следует из автономности дифференциального уравнения.

Очевидно, что $E=F=\mathbb{R}$.

Покажем, что у полугруппы трансляций (\ref{oper_sdviga_primer_1}) нет $(\mathbb{R},\mathbb{R})$-аттрактора.

Предположим противное.
Пусть такой аттрактор $A \subset \mathbb{R}$ существует.
Тогда $A$ должно быть инвариантно относительно полугруппы трансляций $S_t$ и компактно в $\R$, следовательно, и замкнуто в $\R$.
Используя запись (\ref{oper_sdviga_primer_1}), легко найти, что
\begin{equation}\label{oper_sdviga_primer_1_proizv}
	\frac{d}{dt}S_t p =
	\left\{
		\begin{array}{ll}
			\frac{-(p-1)^2}{(pt-t+1)^2} < 0, & p > 1
		\\\\
			0,        & p = 0
		\\\\
			\frac{-p^2}{(pt+1)^2} < 0,        & 0 < p \leq 1
		\\\\
			\frac{-p}{1 - pt} > 0,      & p < 0
		\end{array}
	\right.
\end{equation}

Непонятно: нужно ли подробно расписывать дифференцирование?

Так как множество $A$ замкнуто в $\R$, то оно содержит свои минимум и максимум.

Пусть $m = \min A < 0$.
Тогда в силу (\ref{oper_sdviga_primer_1_proizv}) $S_1 m > m$, следовательно, $m \notin S_1 A$ и $S_1 A \neq A$.
Следовательно, $m \geq 0$.
Но тогда и $M = \max A \geq 0$.

Пусть $M > 0$.
В силу (\ref{oper_sdviga_primer_1_proizv}) $S_1 M < M$, следовательно, $M \notin S_1 A$ и $S_1 A \neq A$.

Значит, $M=m=0$, т.е. $A=\{0\}$.
Покажем, что в таком случае множество $A$ не является $(\R,\R)$-притягивающим для полугруппы трансляций $S_t$, а следовательно, не является $(\R,\R)$-аттрактором для полугруппы трансляций $S_t$.

Действительно, для множества $A$ рассмотрим его открытую окрестность $ W = (-\frac12;\frac12)$.
Положим $B_2={2}$.
Очевидно, что $B_2$ ограничено в $\R$.
Тогда из \ref{oper_sdviga_primer_1} и \ref{oper_sdviga_primer_1_proizv} следует, что
$$
	\forall(t>0)[S_t B_2 \subset [1; 2]],
$$
но $[1;2]\cap W = \varnothing$, откуда
$$
	\forall(t>0)[S_t B_2 \not\subset W],
$$

т.е. $A$ не является $(\R,\R)$-притягивающим для полугруппы трансляций $S_t$,
а значит, и $(\R,\R)$-аттрактором для $S_t$.

Таким образом, мы доказали, что у полугруппы трансляций, заданной соотношением (\ref{oper_sdviga_primer_1}),
нет $(\R,\R)$-аттрактора.

Покажем теперь, что теория минимальных траекторных аттракторов и глобальных аттракторов
даёт для этого примера более содержательные результаты.


\subsection*{Глобальный аттрактор}

\opred (\cite{Zelenaya}, параграф 3.2, стр. 121-122, опр. 3.2.1)

Множество  $\mathcal{A} \subset E $ называется глобальным аттрактором (в $E_0$) для пространства траекторий $\mathcal{H}^+$, если:
\begin{enumerate}
	\item)
		$\mathcal{A}$ компактно в $E_0$ и ограничено в $E$;
	\item)
		$
			\forall(K>0)\forall(X \subset \mathcal{H}^+ \cap B_{L_{\infty}\left( \mathbb{R}_+; E \right)}(0,K))
				\left[
					h_{E_0}(X(t),\mathcal{A}) \xrightarrow[t\to \infty]{}{0}
				\right],
		$
		т.е. сечения любого ограниченного в $L_{\infty}\left( \mathbb{R}_+; E \right)$ множества
		с течением времени <<притягиваются>> к $\mathcal{A}$.
	\item)
		$\mathcal{A}$ --- наименьшее по включению множество, удовлетворяющее условиям (1) и (2).
\end{enumerate}


Глобальный аттрактор, как следует из определения, принадлежит фазовому пространству.
Глобальный аттрактор представляет собой набор состояний, к которым система будет стремиться со временем;
все остальные состояния системы, т.е. точки фазового пространства, не входящие в глобальный аттрактор, преходящи;
с учётом того, что в реальных измерениях всегда присутствует некоторая погрешность, тем фактом, что решения могут не достигать глобального аттрактора, а только приближаться к нему, можно пренебречь и изучать только состояния системы, соответствующие глобальному аттрактору.

Покажем, что для рассматриваемого примера глобальным аттрактором является множество $G=\{0; 1\} \subset \R$.
Очевидно, что $G$ компактно и ограничено в $\R$.
Проверим второе условие:
$$
	\forall(K>0)\forall(X \subset \mathcal{H}^+ \cap B_{L_{\infty}\left( \R_+; \R \right)}(0,K))
		\left[
			h_{\R}(X(t),G) \xrightarrow[t\to \infty]{}{0}
		\right].
$$
Действительно, зафиксируем $K>0$.
Тогда $\mathcal{H}^+ \cap B_{L_{\infty}\left( \R_+; \R \right)} = \{ S_t p ~|~ -K \leq p \leq K\} $,
где $S_t p$ --- оператор сдвига, определённый (\ref{oper_sdviga_primer_1}).
Для произвольного $X \subset \mathcal{H}^+ \cap B_{L_{\infty}\left( \R_+; \R \right)}(0,K)$
\begin{multline*}
	h_{\R}(X(t),G) =
	\sup_{r\in X(t)} \inf_{q \in G} | q - r | =
	\sup_{r\in X(t)} \min\{r, r - 1\} \leq
	\sup_{p\in (-K, K)} \min\{|S_t p|, |S_t p - 1|\} =
	\\ =
	\left\{
		\begin{array}{ll}
			\frac{p - 1}{1 + (p-1)t},    & 1 < p < K
		\\\\
			\frac{p}{1 + pt},            & 0 \leq p \leq 1
		\\\\
			\frac{-p}{1 + (-p)t},        & -K < p < 0
		\end{array}
	\right\} \xrightarrow[t\to \infty]{}{0}
\end{multline*}

Покажем теперь, что множество $G$ есть минимальное по включению множество,
удовлетворяющее первым двум условиям из определения глобального аттрактора.
Действительно, его непустыми подмножествами являются только $\{0\}$ и $\{1\}$.
Положим $K=3$.
Рассмотрим множества $X_0 = \{0(t)\}$ и $X_1 = \{S_t 2\} = \{ 1+\frac{1}{1+t} \}$.
Очевидно, что $X_i \subset \mathcal{H}^+ \cap B_{L_{\infty}\left( \R_+; \R \right)}(0,3)$, $i=0,1$.
Но
$$
h_{\R}(X_0(t),\{1\}) = |0 - 1| = 1,
$$
$$
h_{\R}(X_1(t),\{0\}) = \left|1+\frac{1}{1+t} - 0\right| > 1.
$$
Таким образом, $G$ --- мнимальное по включению множество, удовлетворяющее второму пунтку определения глобального аттрактора,
и, следовательно, являетчя глобальным аттрактором исследуемого уравнения.

\subsection*{Минимальный траекторный аттрактор}

Покажем, что минимальный траекторный аттрактор существует, и найдём его.

\opred (\cite{zhidkosti_s_pamyatyu}, опр. 2; \cite{Zelenaya}, параграф 3.2, стр. 122, опр. 3.2.2; предл. 3.2.1)

Множество $P \subset T_+$ называется притягивающим для пространства траекторий $\mathcal{H}^+$ в топологии пространства $C(\mathbb{R}_+; E_0)$,
если для всякого множества $B \subset \mathcal{H}^+$, ограниченного в $L_{\infty}(\mathbb{R}_+;E)$, выполняется условие
\begin{equation}
	h_{E_0}(B(t),T(t)P) \xrightarrow[t\to\infty]{}0
\end{equation}


\opred (\cite{Zelenaya}, параграф 3.2, стр. 124, опр. 3.2.3; \cite{zhidkosti_s_pamyatyu}, опр. 3)

Непустое множество $\mathcal{U}\subset\mathcal{H}^+$ называется траекторным аттрактором в пространстве траекторий $\mathcal{H}^+$ относительно топологии $C(\mathbb{R}_+,E_0)$, если
\begin{enumerate}
	\item)
		$\mathcal{U}$ компактно в $C(\mathbb{R}_+,E_0)$;
	\item)
		$\mathcal{U}$ ограничено в $L_{\infty}(\mathbb{R}_+,E)$;
	\item)
		$\mathcal{U}$ трансляционно инвариантно;
	\item)
		$\mathcal{U}$ является притягивающим для пространства траекторий $\mathcal{H}^+$ в топологии пространства $C(\mathbb{R}_+; E_0)$.
\end{enumerate}

\opred (\cite{zhidkosti_s_pamyatyu}, опр. 3)
Минимальным траекторным аттрактором пространства траекторий $\mathcal{H}^+$ называется такой траекторный аттрактор, который содержится в любом другом траекторном аттракторе.


Покажем наконец, что для уравнения (\ref{primer_iz_statyi}) минимальный траекторный аттрактор имеет вид
\begin{equation}
	P = \{ v_0(t) \equiv 0, v_1(t) \equiv 1\}
\end{equation}

Покажем сначала, что $P$ --- траекторный аттрактор.

Очевидно, что конечное множество $P$ и компактно, и ограничено в любых пространствах.
Трансляционная инвариантность функций, тождественно равных константе, также очевидна.

Для доказательства того, что $P$ --- траекторный аттрактор осталось доказать, что $P$ --- притягивающее множество для пространства траекторий $\mathcal{H}^+$ в топологии пространства $С(\mathbb{R}_+$.

Пусть $Q \subset \mathcal{H}^+$ --- ограниченное в $L_{\infty}(\mathbb{R}_+;E)$ множество.
Тогда
\begin{equation}
	\exists(K>0)\forall(u\in Q)\left[ \| u \|_{L_{\infty}(\mathbb{R}_+;E)} < K\right]
\end{equation}
т.е., с учётом того, что
$$
	\| u \|_{L_{\infty}(\mathbb{R}_+;E)} = \sup_{t\geq 0}|u(t)|
$$
получаем, что
\begin{equation}
	\exists(K>0)\forall(u\in Q)\forall(t \geq 0)\left[ | u (t) | < K\right]
\end{equation}
и, в частности,
\begin{equation}
	\exists(K>0)\forall(u\in Q)\left[ | u (0) | < K\right]
\end{equation}

Положим $K>1$ и зафиксируем  константу $K$.

Пусть решение имеет вид (\ref{primer_p1_solution}).
Тогда в выражении $u_1(t)=\frac{1}{t+\frac{1}{p-1}}+1$  (см. (\ref{primer_zk_1_0})) $1<p<K$ и
\begin{equation}
	|u_1(t) - v_1(t)| =
	|u_1(t) - 1| =
	\left|\left(\frac{1}{t+\frac{1}{p-1}}+1\right) -1\right| =
	\left|\frac{1}{t+\frac{1}{p-1}}\right| \leq
	\left|\frac{1}{t}\right| =
	\frac{1}{t}
	\xrightarrow[t\to+\infty]{}0
\end{equation}

Пусть решение имеет вид (\ref{primer_p2_solution}).

Тогда в выражении $u_2(t)=\frac{1}{t+\frac{1}{p}}$  (см. (\ref{primer_zk_2_0})) $0 < p \leq 1 < K$ и
\begin{equation}
	|u_2(t) - v_0(t)| =
	|u_2(t) - 0| =
	|u_2(t)| =
	\left| \frac{1}{t+\frac{1}{p}} \right| \leq
	\left|\frac{1}{t}\right| =
	\frac{1}{t}
	\xrightarrow[t\to+\infty]{}0
\end{equation}

Случай, когда решение имеет вид $u_3(t) \equiv 0$, тривиален.

Пусть, наконец, решение имеет вид (\ref{primer_p3_solution}).

Тогда в выражении $u_4(t)=-\frac{1}{t-\frac{1}{p}}$  (см. (\ref{primer_zk_4_0})) $-K < p < 0$ и
\begin{equation}
	|u_4(t) - v_0(t)| =
	|u_4(t) - 0| =
	|u_4(t)| =
	\left| -\frac{1}{t-\frac{1}{p}} \right| =
	\left| \frac{1}{t+\frac{1}{-p}} \right| \leq
	\left|\frac{1}{t}\right| =
	\frac{1}{t}
	\xrightarrow[t\to+\infty]{}0
\end{equation}

Заметим, что в данном примере стремление функций $u_1$, $u_2$, $u_3$ и $u_4$ к соответствующим функциям из траекторного аттрактора не зависит от выбора константы $K$: функции мажорируются гиперболами.

Следовательно,
$$
	h_{\mathbb{R}}(Q(t),P(t)) =
	\sup_{u\in Q} \inf_{v_i \in P} |u(t) - v_i(t)| =
	\sup_{u\in Q} \min(|u(t) - v_0(t)|,|u(t) - v_1(t)|) \leq
	\sup_{u\in Q} \frac{1}{t} =
	\frac{1}{t}
	\xrightarrow[t\to+\infty]{}0
$$

Таким образом, множество $P$ действительно является притягивающим для пространства траекторий $\mathcal{H}^+$ уравнения (\ref{primer_iz_statyi}) и, с учётом выше, является траекторным аттрактором.

Покажем, что $P$ --- минимальный траекторный аттрактор.
Предположим противное.
Тогда либо $P_0 =\{v_0(t) \equiv 0\}$, либо $P_1 =\{v_1(t) \equiv 1\}$ также является траекторным аттрактором.
Но $P_1$ --- не траекторный аттрактор, т.к. для решения $u_3(t) \equiv 0$ имеем
\begin{equation}
	h_{\mathbb{R}}(\{u_3(t) \equiv 0 \},P_1(t)) =
	|u_3(t) - v_1(t)| =
	|0-1| =
	1
	%\not{\xrightarrow[t\to+\infty]{}0}
\end{equation}

Аналогично траекторным аттрактором не является $P_0$.
Действительно, для решения вида $u_1(t)=\frac{1}{t+\frac{1}{p-1}}+1, p > 1$ имеем

\begin{equation}
	h_{\mathbb{R}}(\{u_1(t)\},P_0(t)) =
	|u_1(t) - v_0(t)| =
	|u_1(t) - 0| =
	|u_1(t)| =
	\left| \frac{1}{t+\frac{1}{p-1}}+1 \right| =
	\frac{1}{t+\frac{1}{p-1}}+1 >	1
	%\not{\xrightarrow[t\to+\infty]{}0}
\end{equation}

Следовательно, траекторный аттрактор $P$ --- минимальный.
Примечательно, что функция $v_1(t) \equiv 1$ принадлежит минимальному траекторному аттрактору,
но не принадлежит пространству траекторий.




\addcontentsline{toc}{chapter}{Литература}
\begin{thebibliography}{99}

\bibitem{Vorotnikov} Topological Approximation Methods for Evolutionary Problems of Nonlinear Hydrodynamics / Victor G. Zvyagin, Dmitry A. Vorotnikov. Walter de Gruyter, Berlin, New York. 2008.

\bibitem{Kondratyev} Аттракторы уравнений неньютоновской гидродинамики / В. Г. Звягин, С. К. Кондратьев. – Успехи математических наук, 2014, сентябрь-октябрь, т. 69, вып 5 (419). – 76с.

\bibitem{Fursikov} Оптимальное управление распределёнными системами. Теория и приложения / А. В. Фурсиков. Новосибирск. Научная книга, 1999.

\bibitem{Zelenaya} Аттракторы для уравнений моделей движения вязкоупругих сред : учебное пособие / В.\,Г.\,Звягин, С.\,К.\,Кондратьев~; Воронежский государственный университет. -- Воронеж : Издательско-полиграфический центр Воронежского государственного университета, 2010. --- 266 с.

\bibitem{zhidkosti_s_pamyatyu} Аттракторы слабых решений регуляризованной системы уравнений движения жидких сред с памятью / В.\,Г.\,Звягин, С.\,К.\,Кондратьев~; Известия вузов: Математика, 2011, № 8 --- c. 86–89

\end{thebibliography}

\end{document}
