\chapter*{Введение}
\addcontentsline{toc}{chapter}{Введение}
В настоящей работе рассматриваются траекторные аттракторы и аттракторы динамических систем.
Основная цель работы --- изучить примеры минимальных траекторных аттракторов,
не лежащих внутри соответствующих пространств траекторий.

Подробно рассматриваются два примера таких аттракторов: пример минимального траекторного аттрактора,
частично лежащего в пространстве траекторий, ранее описанный проф. В.Г. Звягиным в \cite{Zelenaya},
и пример минимального траекторного аттрактора, целиком лежащего вне пространства траекторий,
впервые изучаемый в данной работе.

Оба примера порождаются обыкновенными дифференциальными уравнениями с кусочно-гладкой правой частью.
