\input{lib/packs}
\LARGE
\begin{document}
\input{lib/macro}

Пример системы, траекторный аттрактор которой частично лежит вне пространства траекторий,
и у которой при этом существует аттрактор в смысле динамической системы.

Первый пример:

Полярная система координат на плоскости.
Для удобства считаем $\varphi = 0$ при $r=0$.

\begin{equation}
	\left\lbrack
		\begin{array}{ll}
			\left\{
				\begin{array}{l}
					\varphi ' = \varphi
				\\
					r' = 0
				\end{array}
			\right.
			& , \mbox{~если~} r = 1
			\\\\
			\left\{
				\begin{array}{l}
					\varphi ' = 0
				\\
					r' = |1-r| \cdot (1-r)
				\end{array}
			\right.
			& , \mbox{~если~} r \neq 1
		\end{array}
	\right.
\end{equation}

Аттрактор динамической системы --- окружность $r = 1$.
Она инвариантна, точка $(1; 0)$ честно порождает константу
(при конце оборота, где должно быть $2\pi$, получается $\varphi' = \varphi = 0$).

МТА --- забор из констант $(1;\varphi_0)$.
МТА частично лежит в пространстве траекторий
(перечесение по единственному элементу $(1; 0)$).

Ценность в том, что здесь есть АДС и при этом среди траекторий, выходящих из АДС, нет периодичных.

\end{document}
