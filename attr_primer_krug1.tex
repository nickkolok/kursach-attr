\documentclass[a4paper,14pt]{report} %размер бумаги устанавливаем А4, шрифт 12пунктов
\usepackage[T2A]{fontenc}
\usepackage[utf8]{inputenc}
\usepackage[english,russian]{babel} %используем русский и английский языки с переносами
\usepackage{amssymb,amsfonts,amsmath,mathtext,cite,enumerate,float,amsthm} %подключаем нужные пакеты расширений
\usepackage[pdftex,unicode,colorlinks=true,linkcolor=blue]{hyperref}
\usepackage{indentfirst} % включить отступ у первого абзаца
\usepackage[dvips]{graphicx} %хотим вставлять рисунки?
\graphicspath{{illustr/}}%путь к рисункам

\makeatletter
\renewcommand{\@biblabel}[1]{#1.} % Заменяем библиографию с квадратных скобок на точку:
\makeatother %Смысл этих трёх строчек мне непонятен, но поверим "Запискам дебианщика"

\usepackage{geometry} % Меняем поля страницы. 
\geometry{left=1cm}% левое поле
\geometry{right=1cm}% правое поле
\geometry{top=1cm}% верхнее поле
\geometry{bottom=2cm}% нижнее поле

\renewcommand{\theenumi}{\arabic{enumi}}% Меняем везде перечисления на цифра.цифра
\renewcommand{\labelenumi}{\arabic{enumi}}% Меняем везде перечисления на цифра.цифра
\renewcommand{\theenumii}{.\arabic{enumii}}% Меняем везде перечисления на цифра.цифра
\renewcommand{\labelenumii}{\arabic{enumi}.\arabic{enumii}.}% Меняем везде перечисления на цифра.цифра
\renewcommand{\theenumiii}{.\arabic{enumiii}}% Меняем везде перечисления на цифра.цифра
\renewcommand{\labelenumiii}{\arabic{enumi}.\arabic{enumii}.\arabic{enumiii}.}% Меняем везде перечисления на цифра.цифра


\LARGE
\begin{document}
\newcommand{\pp}{Предположим противное}
%\newcommand{\pp}{{\LARП\!\!\!\!п~}}
\newcommand{\dokvo}{\paragraph{Доказательство.}}
\newcommand{\dokno}{\textbf {Доказано.}}
\newcommand{\neobh}{\paragraph{Необходимость.}}
\newcommand{\dost }{\paragraph{Достаточность.}}
\newcommand{\opred}{\paragraph{Определение.}}
\newcommand{\mnemo}{\paragraph{Мнемоника.}}
\newcommand{\N}{\mathbb{N}}
\newcommand{\Z}{\mathbb{Z}}
\newcommand{\Q}{\mathbb{Q}}
\newcommand{\R}{\mathbb{R}}
\newcommand{\one}[1]{\mathbb{I}_{#1}}
\renewcommand{\C}{\mathbb{C}}
\newcommand{\Beta}{B}%Костыль, а что поделать?
\newcommand{\Rn}{$\mathbb{R}^n~$}
\newcommand{\Rm}{$\mathbb{R}^m~$}
\renewcommand{\epsilon}{\varepsilon}
\renewcommand{\geq}{\geqslant}
\renewcommand{\leq}{\leqslant}
\newcommand{\fXR}{Пусть $X \subset \R, f:X \to \R$ }
\newcommand{\fXRx}{\fXR, $x_0$ - предельная точка $X$ }
\newcommand{\sgn}{\mathrm{sgn}~}
\newcommand{\nid}{\Leftrightarrow}
\newcommand{\intl}{\int\limits}
\newcommand{\suml}{\sum\limits}
\newcommand{\Models}{|\!\!\!=\!\!\!|}
\newcommand{\Rightleftarrow}{\Leftrightarrow}

\newcommand{\xI}{{\vec{\xi}}}%Костыль для тервера, очень уж там часто встречается
\newcommand{\calF}{\mathcal{F}}
\newcommand{\calB}{\mathcal{B}}
\newcommand{\GOFP}{$G \sim \left<\Omega,\calF,P\right>$}

\newenvironment{zamena}[1][c]{=\left<\begin{array}{#1}}{\end{array}\right>=}

\newtheorem{theorem}{Теорема}[section]
\newenvironment{teorema}[1][{}]{\begin{theorem}{#1}\upshape}{\end{theorem}}

\theoremstyle{definition}

\newtheorem{zamech}{Замечание}[section]
\newtheorem{primer}{Пример}[section]
\newtheorem{opr}{Определение}[section]

\newtheorem{sledstvie}{Следствие}[theorem]
\newtheorem{utverzhd}[theorem]{Утверждение}
\newtheorem{lemma}[theorem]{Лемма}

\long\def\comment{}


Пример системы, траекторный аттрактор которой частично лежит вне пространства траекторий,
и у которой при этом существует аттрактор в смысле динамической системы.

Первый пример:

Полярная система координат на плоскости.
Для удобства считаем $\varphi = 0$ при $r=0$.

\begin{equation}
	\left\lbrack
		\begin{array}{ll}
			\left\{
				\begin{array}{l}
					\varphi ' = \varphi
				\\
					r' = 0
				\end{array}
			\right.
			& , \mbox{~если~} r = 1
			\\\\
			\left\{
				\begin{array}{l}
					\varphi ' = 0
				\\
					r' = |1-r| \cdot (1-r)
				\end{array}
			\right.
			& , \mbox{~если~} r \neq 1
		\end{array}
	\right.
\end{equation}

Аттрактор динамической системы --- окружность $r = 1$.
Она инвариантна, точка $(1; 0)$ честно порождает константу
(при конце оборота, где должно быть $2\pi$, получается $\varphi' = \varphi = 0$).

МТА --- забор из констант $(1;\varphi_0)$.
МТА частично лежит в пространстве траекторий
(перечесение по единственному элементу $(1; 0)$).

Ценность в том, что здесь есть АДС и при этом среди траекторий, выходящих из АДС, нет периодичных.

Второй пример:

СК та же.

\begin{equation}
	\left\lbrack
		\begin{array}{ll}
			\left\{
				\begin{array}{l}
					\varphi ' = 1
				\\
					r' = 0
				\end{array}
			\right.
			& , \mbox{~если~} r = 1
			\\\\
			\left\{
				\begin{array}{l}
					\varphi ' = 0
				\\
					r' = |1-r| \cdot (1-r)
				\end{array}
			\right.
			& , \mbox{~если~} r \neq 1
		\end{array}
	\right.
\end{equation}

Т.~е. на окружности крутятся спиральки, а всё остальное к этой окружности стягивается.

МТА --- забор из констант $(1;\varphi_0)$ и цилиндр из спиралек на окружности, которые решения.
МТА частично лежит в пространстве траекторий
(перечесение по забору).

Аттрактор динамической системы --- окружность $r = 1$.
Она инвариантна, просто крутится на месте.

\end{document}
